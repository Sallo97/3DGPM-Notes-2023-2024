\appendix

\chapter{Elemental Geometry Concepts}

\section{Topological Space}
In mathematics, a topological space is, roughly speaking, a geometrical space in which closeness is defined but cannot necessarily be measured by a numeric distance. More specifically, a topological space is a set whose elements are called points, along with an additional structure called a topology, which can be defined as a set of neighbourhoods for each point that satisfy some axioms formalizing the concept of closeness. It is the most general type of a mathematical space that allows for the definition of limits, continuity, and connectedness. Common types of topological spaces include Euclidean spaces, metric spaces and manifolds. 

\section{Polytope}
In elementary geometry, a polytope is a geometric object with flat sides (faces). Polytopes are the generalization of three-dimensional polyhedra to any number of dimensions. Polytopes may exist in any general number of dimensions n as an n-dimensional polytope or n-polytope. For example, a two-dimensional polygon is a 2-polytope and a three-dimensional polyhedron is a 3-polytope.

\section{Curve}
In mathematics, a curve (also called a curved line in older texts) is an object similar to a line, but that does not have to be straight.\par
Formally a curve is the image of an interval to a topological space by a continuous function. In some contexts, the function that defines the curve is called a parametrization, and the curve is a parametric curve. In this article, these curves are sometimes called topological curves to distinguish them from more constrained curves such as differentiable curves. This definition encompasses most curves that are studied in mathematics; notable exceptions are level curves (which are unions of curves and isolated points), and algebraic curves (see below). Level curves and algebraic curves are sometimes called implicit curves, since they are generally defined by implicit equations. 

\section{Convex Hull}
A convex hull (also known as convex envelope or convex closure) of a shape is the smallest convex set that contains it. The convex hull may be defined either as the intersection of all convex sets containing a given subset of a Euclidean space, or equivalently as the set of all convex combinations of points in the subset. For a bounded subset of the plane, the convex hull may be visualized as the shape enclosed by a rubber band stretched around the subset.

\section{D-simplex}
A d-simplex is the convex hull of d+1 point that are linearly dependent in d dimensions.

\section{Plane}
In mathematics, a plane is a two-dimensional space or flat surface that extends indefinitely. A plane is the two-dimensional analogue of a point (zero dimensions), a line (one dimension) and three-dimensional space. 

\section{Topological Characterization}
How the elements are combinatorially connected.

\section{Geometric Characteriation}
Where the vertices are actually placed in space.

\section{Open Set}
An open set is a set that, along with every point P, contains all points that are sufficiently near to P (that is, all points whose distance to P is less than some value depending on P).\par
More generally, an open set is a member of a given collection of subsets of a given set, a collection that has the property of containing every union of its members, every finite intersection of its members, the empty set, and the whole set itself. A set in which such a collection is given is called a topological space, and the collection is called a topology. These conditions are very loose, and allow enormous flexibility in the choice of open sets. For example, every subset can be open (the discrete topology), or no subset can be open except the space itself and the empty set (the indiscrete topology).\par
In practice, however, open sets are usually chosen to provide a notion of nearness that is similar to that of metric spaces, without having a notion of distance defined. In particular, a topology allows defining properties such as continuity, connectedness, and compactness, which were originally defined by means of a distance. 

\section{Neighborhood of a point}
The neighborhood of a point is a concept of topological space.\par
Given a topological space $X$ and a point $p \in X$, then a neighbourhood of $p$ is a subset $V$ of $X$ that includes an open set $U$ containing $p$ s.t.:
\begin{equation*}
    p \in U \subseteq V \subseteq X
\end{equation*}